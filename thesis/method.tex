%%% lorem.tex --- 
%% 
%% Filename: lorem.tex
%% Description: 
%% Author: Ola Leifler
%% Maintainer: 
%% Created: Wed Nov 10 09:59:23 2010 (CET)
%% Version: $Id$
%% Version: 
%% Last-Updated: Wed Nov 10 09:59:47 2010 (CET)
%%           By: Ola Leifler
%%     Update #: 2
%% URL: 
%% Keywords: 
%% Compatibility: 
%% 
%%%%%%%%%%%%%%%%%%%%%%%%%%%%%%%%%%%%%%%%%%%%%%%%%%%%%%%%%%%%%%%%%%%%%%
%% 
%%% Commentary: 
%% 
%% 
%% 
%%%%%%%%%%%%%%%%%%%%%%%%%%%%%%%%%%%%%%%%%%%%%%%%%%%%%%%%%%%%%%%%%%%%%%
%% 
%%% Change log:
%% 
%% 
%% RCS $Log$
%%%%%%%%%%%%%%%%%%%%%%%%%%%%%%%%%%%%%%%%%%%%%%%%%%%%%%%%%%%%%%%%%%%%%%
%% 
%%% Code:

\chapter{Method} \label{cha:method}

    \section{Implementation} \label{sec:implementation}

        \subsection{System Overview} \label{sub:system_overview}

        \subsection{Quadric-Based Error Metric} \label{sub:quadric-based_error_metric}

        \subsection{Appearance-Preserving Simplification} \label{sub:appearance-preserving_simplification}

        \subsection{Texture Mapped Progressive Meshing} \label{sub:texture_mapped_progressive_meshing}

    \newpage

    \section{Evaluation} \label{sec:evaluation}

        In order to determine which of these algorithms provide the best performance for a target appearance threshold, an evaluation of the polygon count, computation time, memory usage and rendering time of the simplified mesh is done for each of the implemented solutions. In the results from this step, a series of tables are generated to compare the performance between the algorithms by using a common comparison framework. In this section, we describe this common comparison framework and then show how we can measure each of the parameters.

        In essence, this is done by targeting a certain appearance threshold, tweaking the mesh simplification algorithm's parameters to achieve this threshold, and then measuring the given performance. This gives a universal measure of ``quality'' for all of the algorithms, which would otherwise have different error metrics used for applying the simplification. Since the performance measures are noisy, a total of \(n=20\) samples will be taken. According to \emph{David Lilja}~\cite[p.~50]{lilja2005measuring} the t-student distribution should be used when \(n < 30\), as shown in Section~\ref{sec:measuring_algorithmic_performance}.

        The pack of test meshes that are going to be used in the comparison are a combination of textured models provided by Configura and others taken from the public domain. The exact selection of these is still to be decided, but should include both low- \& high-polygon meshes.

        \subsection{Appearance Preservation} \label{sec:appearance_preservation}
        In order to compare the appearance preservation of the mesh simplification algorithms, the image-metric explained in section~\ref{sec:metrics_for_appearance_preservation} is used. It is useful since it can compare the difference of any two meshes, therefore, it does not depend on the algorithm used.

        For both the original mesh and a simplified mesh, 24 images with resolution $512 \times 512$ is rendered with a simple renderer based on \emph{OpenGL}. The camera is placed at the vertices of a rhombicuboctahedron and is faced towards the center where the mesh is placed. A light source is placed at the camera position. This will make sure that the surface facing the camera will be illuminated.

        The two sets of 24 images each is used to compute the RMS of the simplified mesh with equation~\ref{eq:rms_image_sets}. This RMS value can then be used to compare how well the algorithms perform. 
        
        \subsection{Polygon Count} \label{sec:polygon_count}

        \subsection{Computation Time} \label{sec:computation_time}

        An important property of a mesh simplification algorithm is the time it takes to simplify a full-resolution mesh to a lower-resolution mesh. While this doesn't impact the run-time of the mesh (that is accounted for by the rendering time, measured in Section~\ref{sec:rendering_time}), it is still important to reduce it as much as possible. This is especially true if the LoD is dynamically generated at run-time, but it is also important since many more meshes can be simplified per time unit (important if a simplifier is to be provided as a service, as \emph{Simplygon Linköping} does).

        In order to compare the execution time of the different algorithms, they all target the same appearance thresholds as specified in Section~\ref{sec:appearance_preservation} by tweaking the parameters unique to each algorithm. We then measure the time it takes for the simplification algorithm to execute when using these parameters, in other words, simply by: \(time_\mathcal{A} = end_\mathcal{A} - start_\mathcal{A}\) for an algorithm \(\mathcal{A}\). Of course, this measurement is done several times to account for noise. After calculating the mean \(\bar{x}\) and the standard-deviation \(s\), one can find the confidence interval \([a, b]\) of the execution time by the equations shown in Section~\ref{sec:measuring_algorithmic_performance} with 20 degrees of freedom and \(\alpha = 5 \%\).

        \subsection{Memory Usage} \label{sec:memory_usage}

        Another important performance property of the algorithm is the accumulated memory used when simplifying the mesh. Depending on the size of the mesh in triangles, the algorithm could consume large amounts of memory (and might not even fit in the primary memory in some cases). It is therefore important to compare the simplification algorithms to determine those which are suited for optimizing large triangle meshes and those that aren't. Since this measure will always be deterministic (at least for the method we use to measure it), there is no need to apply any statistical measures.

        \begin{lstlisting}[language=bash]
(*\textbf{valgrind}*) --tool=massif (*\textit{./simplify --algorithm=<algorithm> <input-mesh> <output-mesh>}*)
        \end{lstlisting}

        \subsection{Rendering Time} \label{sec:rendering_time}

%%%%%%%%%%%%%%%%%%%%%%%%%%%%%%%%%%%%%%%%%%%%%%%%%%%%%%%%%%%%%%%%%%%%%%
%%% method.tex ends here

%%% Local Variables: 
%%% mode: latex
%%% TeX-master: "thesis"
%%% End: 
