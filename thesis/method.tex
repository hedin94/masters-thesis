%%% lorem.tex --- 
%% 
%% Filename: lorem.tex
%% Description: 
%% Author: Ola Leifler
%% Maintainer: 
%% Created: Wed Nov 10 09:59:23 2010 (CET)
%% Version: $Id$
%% Version: 
%% Last-Updated: Wed Nov 10 09:59:47 2010 (CET)
%%           By: Ola Leifler
%%     Update #: 2
%% URL: 
%% Keywords: 
%% Compatibility: 
%% 
%%%%%%%%%%%%%%%%%%%%%%%%%%%%%%%%%%%%%%%%%%%%%%%%%%%%%%%%%%%%%%%%%%%%%%
%% 
%%% Commentary: 
%% 
%% 
%% 
%%%%%%%%%%%%%%%%%%%%%%%%%%%%%%%%%%%%%%%%%%%%%%%%%%%%%%%%%%%%%%%%%%%%%%
%% 
%%% Change log:
%% 
%% 
%% RCS $Log$
%%%%%%%%%%%%%%%%%%%%%%%%%%%%%%%%%%%%%%%%%%%%%%%%%%%%%%%%%%%%%%%%%%%%%%
%% 
%%% Code:

\chapter{Method} \label{cha:method}

    \section{Implementation} \label{sec:implementation}

        \subsection{System Overview} \label{sub:system_overview}

        \subsection{Quadric-Based Error Metric} \label{sub:quadric-based_error_metric}

        \subsection{Appearance-Preserving Simplification} \label{sub:appearance-preserving_simplification}

        \subsection{Texture Mapped Progressive Meshing} \label{sub:texture_mapped_progressive_meshing}

    \newpage

    \section{Evaluation} \label{sec:evaluation}

        In order to determine which of these algorithms provide the best performance for a target appearance threshold, an evaluation of the polygon count, computation time, memory usage and rendering time of the simplified mesh is done for each of the implemented solutions. In the results from this step, a series of tables are generated to compare the performance between the algorithms by using a common comparison framework. In this section, we describe this common comparison framework and then show how we can measure each of the parameters.

        In essence, this is done by targeting a certain appearance threshold, tweaking the mesh simplification algorithm's parameters to achieve this threshold, and then measuring the given performance. This gives a universal measure of ``quality'' for all of the algorithms, which would otherwise have different error metrics used for applying the simplification. Since the performance measures are noisy, a total of \(n=20\) samples will be taken. According to \emph{David Lilja}~\cite[p.~50]{lilja2005measuring} the t-student distribution should be used when \(n < 30\), as shown in Section~\ref{sec:measuring_algorithmic_performance}.

        The pack of test meshes that are going to be used in the comparison are a combination of textured models provided by Configura and others taken from the public domain. The exact selection of these is still to be decided, but should include both low- \& high-polygon meshes.

        \subsection{Appearance Preservation} \label{sub:appearance_preservation}
        In order to compare the appearance preservation of the mesh simplification algorithms, the image-metric explained in section~\ref{sec:metrics_for_appearance_preservation} is used. It is useful since it can compare the difference of any two meshes, therefore, it does not depend on the algorithm used.

        For both the original mesh and a simplified mesh, 24 images with resolution $512 \times 512$ is rendered with a simple renderer based on \emph{OpenGL}. The camera is placed at the vertices of a rhombicuboctahedron and is faced towards the center where the mesh is placed. A light source is placed at the camera position. This will make sure that the surface facing the camera will be illuminated.

        The two sets of 24 images each is used to compute the RMS of the simplified mesh with equation~\ref{eq:rms_image_sets}. This RMS value can then be used to compare how well the algorithms perform. 
        
        \subsection{Polygon Count} \label{sub:polygon_count}
        Concerning research question 3, the appearance preservation for a specific target polygon count needs to be measured. Therefore, the simplification algorithms is tasked to simplify until the target polygon count is reached. When it is reached, the image-metric is used to measure how well the appearance is preserved. Measurements will be performed for multiple target polygon counts. 

        \subsection{Computation Time} \label{sub:computation_time}

        \subsection{Memory Usage} \label{sub:memory_usage}

        \subsection{Rendering Time} \label{sub:rendering_time}

%%%%%%%%%%%%%%%%%%%%%%%%%%%%%%%%%%%%%%%%%%%%%%%%%%%%%%%%%%%%%%%%%%%%%%
%%% method.tex ends here

%%% Local Variables: 
%%% mode: latex
%%% TeX-master: "thesis"
%%% End: 
