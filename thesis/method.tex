%%% lorem.tex --- 
%% 
%% Filename: lorem.tex
%% Description: 
%% Author: Ola Leifler
%% Maintainer: 
%% Created: Wed Nov 10 09:59:23 2010 (CET)
%% Version: $Id$
%% Version: 
%% Last-Updated: Wed Nov 10 09:59:47 2010 (CET)
%%           By: Ola Leifler
%%     Update #: 2
%% URL: 
%% Keywords: 
%% Compatibility: 
%% 
%%%%%%%%%%%%%%%%%%%%%%%%%%%%%%%%%%%%%%%%%%%%%%%%%%%%%%%%%%%%%%%%%%%%%%
%% 
%%% Commentary: 
%% 
%% 
%% 
%%%%%%%%%%%%%%%%%%%%%%%%%%%%%%%%%%%%%%%%%%%%%%%%%%%%%%%%%%%%%%%%%%%%%%
%% 
%%% Change log:
%% 
%% 
%% RCS $Log$
%%%%%%%%%%%%%%%%%%%%%%%%%%%%%%%%%%%%%%%%%%%%%%%%%%%%%%%%%%%%%%%%%%%%%%
%% 
%%% Code:

\chapter{Method} \label{cha:method}

    \section{Implementation} \label{sec:implementation}

        \subsection{System Overview} \label{sub:system_overview}

        \subsection{Quadric-Based Error Metric} \label{sub:quadric-based_error_metric}

        \subsection{Appearance-Preserving Simplification} \label{sub:appearance-preserving_simplification}

        \subsection{Texture Mapped Progressive Meshing} \label{sub:texture_mapped_progressive_meshing}

    \section{Evaluation} \label{sec:evaluation}

        \subsection{Appearance Preservation} \label{sub:appearance_preservation}
        In order to compare the appearance preservation of the mesh simplification algorithms, the image-metric explained in section~\ref{sec:metrics_for_appearance_preservation} is used. It is useful since it can compare the difference of any two meshes, therefore, it does not depend on the algorithm used.

        For both the original mesh and a simplified mesh, 24 images with resolution $512 \times 512$ is rendered with a simple renderer based on \emph{OpenGL}. The camera is placed at the vertices of a rhombicuboctahedron and is faced towards the center where the mesh is placed. A light source is placed at the camera position. This will make sure that the surface facing the camera will be illuminated.

        The two sets of 24 images each is used to compute the RMS of the simplified mesh with equation~\ref{eq:rms_image_sets}. This RMS value can then be used to compare how well the algorithms perform. 
        
        \subsection{Computation Time} \label{sub:computation_time}

        \subsection{Memory Usage} \label{sub:memory_usage}

        \subsection{Rendering Time} \label{sub:rendering_time}

%%%%%%%%%%%%%%%%%%%%%%%%%%%%%%%%%%%%%%%%%%%%%%%%%%%%%%%%%%%%%%%%%%%%%%
%%% method.tex ends here

%%% Local Variables: 
%%% mode: latex
%%% TeX-master: "thesis"
%%% End: 
