%%% Abstract.tex --- 
%% 
%% Filename: Abstract.tex
%% Description: 
%% Author: Ola Leifler
%% Maintainer: 
%% Created: Thu Oct 14 13:34:11 2010 (CEST)
%% Version: $Id$
%% Version: 
%% Last-Updated: Tue Dec  1 15:19:52 2015 (+0100)
%%           By: Ola Leifler
%%     Update #: 4
%% URL: 
%% Keywords: 
%% Compatibility: 
%% 
%%%%%%%%%%%%%%%%%%%%%%%%%%%%%%%%%%%%%%%%%%%%%%%%%%%%%%%%%%%%%%%%%%%%%%
%% 
%%% Commentary: 
%% 
%% 
%% 
%%%%%%%%%%%%%%%%%%%%%%%%%%%%%%%%%%%%%%%%%%%%%%%%%%%%%%%%%%%%%%%%%%%%%%
%% 
%%% Change log:
%% 
%% 
%% RCS $Log$
%%%%%%%%%%%%%%%%%%%%%%%%%%%%%%%%%%%%%%%%%%%%%%%%%%%%%%%%%%%%%%%%%%%%%%
%% 
%%% Code:

To decrease the rendering time of a mesh, Level of Detail can be generated by reducing the number of polygons based on some geometrical error. While this works well for most meshes, it is not suitable for meshes with an associated texture atlas. By iteratively collapsing edges based on an extended version of Quadric Error Metric taking both spatial and texture coordinates into account, textured meshes can also be simplified.
%
Results show that constraining edge collapses in the seams of a mesh give poor geometrical appearance when it is reduced to a few polygons. By allowing seam edge collapses and by using a pull-push algorithm to fill areas located outside the seam borders of the texture atlas, the appearance of the mesh is better preserved.

%%%%%%%%%%%%%%%%%%%%%%%%%%%%%%%%%%%%%%%%%%%%%%%%%%%%%%%%%%%%%%%%%%%%%%
%%% Abstract.tex ends here


%%% Local Variables: 
%%% mode: latex
%%% TeX-master: "thesis"
%%% End: 
