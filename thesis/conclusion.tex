\chapter{Conclusion} \label{cha:conclusion}
%In this final chapter the conclusions of the thesis is given.

The purpose of this thesis was to find and integrate a suitable mesh simplification scheme that can be used to generated LoD:s of textured meshes. Several alternatives exist for simplification of a mesh while preserving the textured appearance. By extending quadric based error metrics into a higher dimension both the geometry and texture of the mesh can be preserved. To handle discontinuities in the mesh parameterization a mesh representation where vertices can be associated with multiple texture coordinates were used.

To further improve the appearance a pull-push algorithm was implemented that fill in empty areas in a texture atlas. As shown in the thesis this improvement enhance the quality in the seam of the textured mesh.

\section{Future Work}
As discussed the parallelization with OpenMP did not give much of a speed gain since threads have to wait for write access. A possible solution is to first only compute the face quadrics and not adding them to the vertices. This would remove the write access wait time allowing the threads to execute freely. When all face quadrics have been computed they should be added to the correct vertices. By doing the parallelization per vertex multiple threads can access the face quadrics at a time since they only read the value. Since all these computations now have been made independent of each other an implementation on the GPU would be a good candidate for a speed gain. At least for the initialization phase.

For now, tests have only been made with simplification of meshes where only geometry and texture coordinates. Another attribute that \emph{may} be equally important is the normals of the mesh. It would be interesting to see how the luminance error as well as the sampled geometric error would be affected when normals is also considered. Since the simplification uses general QEM computations can be done in any dimension, thus, extending it to consider normals would be simple. 
